\section{Local Interest Points}

An example of a good local interest point includes the following characteristics:

\begin{itemize}
    \itemsep0em
    \item Invariance to brightness change (local change and global change)
    \item Sufficient texture variation in the local neighbourhood
    \item Incariance to changes between the angle/position of the scene to the camera.
\end{itemize}

\subsection{Finding Local Interest Points}

There are two main methods of detecting local interest points:

\begin{itemize}
    \itemsep0em
    \item Corner Detection - Harris and Stephens
    \item Blob Detection - Difference of Gaussian Extrema
\end{itemize}

\subsubsection{Harris and Stephens Corner Detector}
We look through a small window (frame) of the image pixels, and search for corners. If we shift that window by a small amount, a large change in intensity should be present.

\\

\noindent We measure the \textit{weighted average change in intensity between a window and a shifted version by $(\delta x, \delta y)$ of that window}

\begin{equation}
    E(x,y) = \sum_{W} f(x_{i},y_{i}) [ I(x_{i},y_{i}) - I(x_{i} + \delta x,y_{i} + \delta y) ]^2
\end{equation}

\noindent Lets use the taylor expansion, with the first order terms:

\begin{equation}
    I(x_{i} + \delta x,y_{i} + \delta y) \approx I(x_{i},y_{i}) + [ I_{x}(x_{i},y_{i}) \;\;\: I_{y}(x_{i},y_{i})  ] \begin{bmatrix} \delta x \\ \delta y \end{bmatrix}
\end{equation}

\noindent We can substitute this into the original equation, and simplify:

\begin{align*}
    E(x,y) &= \sum_{W}[ I(x_{i},y_{i}) - I(x_{i} + \delta x,y_{i} + \delta y) ]^2 \\
    &= \sum_{W} (I(x_{i}, y_{i}) - I(x_{i},y_{i}) - [ I_{x}(x_{i},y_{i}) \;\;\: I_{y}(x_{i},y_{i})  ] \begin{bmatrix} \delta x \\ \delta y \end{bmatrix})^2 \\
    &= \sum_{W}([ I_{x}(x_{i},y_{i}) \;\;\: I_{y}(x_{i},y_{i})  ] \begin{bmatrix} \delta x \\ \delta y \end{bmatrix})^2 \\
    &= [\delta x \;\;\; \delta y] \begin{bmatrix} \sum_{w} (I_{x}(x_{i},y_{i}))^2 & \sum_{w} I_{x}(x_{i},y_{i})(I_{y}(x_{i},y_{i}) \\[0.2cm] \sum_{w} I_{x}(x_{i},y_{i})(I_{y}(x_{i},y_{i}) & \sum_{w} (I_{y}(x_{i},y_{i}))^2  \end{bmatrix}\begin{bmatrix} \delta x \\ \delta y \end{bmatrix} \\
    &= \begin{bmatrix} \delta x & \delta y \end{bmatrix}\textbf{M}\begin{bmatrix} \delta x \\ \delta y \end{bmatrix}
\end{align*}

\subsubsection{Structure Tensor}

The square symmetrix matrix M is the called the Structure Tensor or the Second Moment matrix.

\begin{equation}
    M = \begin{bmatrix} \sum_{w} (I_{x}(x_{i},y_{i}))^2 & \sum_{w} I_{x}(x_{i},y_{i})(I_{y}(x_{i},y_{i}) \\[0.2cm] \sum_{w} I_{x}(x_{i},y_{i})(I_{y}(x_{i},y_{i}) & \sum_{w} (I_{y}(x_{i},y_{i}))^2  \end{bmatrix}
\end{equation}

\subsubsection{Response Function}

We can compute the eigenvalues directly, or use the corner response function in terms of the determinant and trace of M.

\begin{align}
    det(M) = M_{00}M_{11} - M_{01}M{10} = M_{00}M_{11}-M_{10}^2 = \lambda_{1}\lambda_{2} \\
    trace(M) = M_{00}+M_{11} = \lambda_{1}+\lambda_{2} \\
    R = det(M) - k\times trace(M)^2
\end{align}

\subsubsection{Application}

We need to differentiate an image, we have the Sobel operator for this. The result from convolution with the x and y template can be used to calculate the structure tensor.

\subsubsection{Multi-Scale}

We define a Gaussian scale space with a fixed \textit{set of scales}, and compute the corner response function at every pixel of each scale, and keep only those with a response above a certain threshold.
\subsection{Scale}

Scale is a problem in computer vision, since as an object gets closer it gets larger with more detail. If we are using a technique with fixed size processing windows, this is a problem.

\subsubsection{Scale Space Theory}
Scale space theory is a formal framework for handling the scale problem.

\begin{itemize}
    \itemsep0em
    \item The image is represented by a series of increasingly smoothed/blurred images parameterised by a scale parameter t.
    \item t represents the amount of smoothing
    \item \textbf{Key Notion: Image structures smaller than $\sqrt{t}$ have been smoothed away at scale t.}
\end{itemize}

\subsubsection{Gaussian Scale Space}
Formally, Gaussian scale space is defined by the following equation:

\begin{equation}
    L(∙,∙;t) = g(∙,∙;t) * f(∙,∙)
\end{equation}

$∙,∙$ implies that $∙$ represents a variable which the convolution is over.

\noindent $t \geq 0$ and,

\begin{equation}
    g(x,y;t) = \frac{1}{2\pi t}e^{-\frac{ (x^2 + y^2)}{2t}}
\end{equation}

$t=\sigma^2$, which means t represents the variance of the Gaussian smoothing filter.

\subsubsection{Nyquist-Shannon Sampling Theorem}

If we filter a signal with a low pass filter that halves the frequency content, you can also half the sampling rate without loss of information.

\begin{quote}
    \textit{If a function $x(t)$ contains no frequencies higher than B$Hz$, it is completely determined by giving its ordinates at a series of points spaced $\frac{1}{2B}$ seconds apart.}
\end{quote}

This leads us to the \textbf{Gaussian Pyramid}. Where, every time we double $t$ in the scale space, we can half the image size without a loss of information. This is faster time complexity wise and uses less memory.

\subsection{Blob Detection}

We recall that the \textit{Laplacian of Gaussian} is the second derivative of a Gaussian function. If we find a local minima or maxima, we get a blob detector. \\



\noindent We can define normalised scale space laplacian of gaussian as:

\begin{equation}
    \Delta_{norm}^{2}L(x,y;t) = t(L_{xx}+L_{yy})
\end{equation}

\noindent By finding extrema of this function in scale space, you can find \textit{blobs} at their representative scale ($\sqrt{2t}$

\begin{quote}
     Very useful property: if a blob is detected at $(x_0, y_0; t_0)$ in an image, then under a scaling of that image by a factor s, the same blob would be detected at ($sx_0, sy_0; s^{2}t_0)$ in the scaled image.
\end{quote}

\noindent Its computationally expensive to build a laplacian of gaussian scale space. The following approximation can be made, the \textbf{Difference of Gaussians}:

\begin{equation}
    \Delta_{norm}^2 L(x,y;t) \approx \frac{t}{\Delta t}(L(x,y;t+\Delta t) - L(x,y;t-\Delta t))
\end{equation}

\noindent The laplacian of gaussian scale space can be built from subtracting adjacent scales of a Gaussian scale space, we can build a difference of gaussian pyramid.

\begin{itemize}
    \itemsep0em
    \item An oversampled pyramid as there are multiple images between a doubling of scale.
    \item Images between a doubling of scale ar an octave.
\end{itemize}