\section{Building Machines That See}

\subsection{Key Terms in Designing CV systems}
Simply we can describe the key elements of a CV system by the following terminology.
\begin{itemize}
    \itemsep0em
    \item robust and repeatable - what we want
    \item invariant - designed to be invariant
    \item constraints - what you apply to make it work
\end{itemize}

\subsubsection{Robustness}
The vision system must be robust to changes in its environment. \\
\noindent \textit{changes in lighting, angle or position of camera.}

\subsubsection{Repeatability}
Repeatability is a measure of robustness. The system must work consistently the same, regardless of environmental changes.

\subsubsection{Invariance}
Invariance to environmental factors helps to achieve robustness and repeatability. Hardware and software can designed to be invariant to certain environmental changes.

\subsubsection{Constraints}
Constraints are what you apply to the hardware, to make the system work in a repeatable and robust fashion. We can constrain a system by putting it in a box so there cannot be an illumination changes. \\

\noindent \textbf{Software Constraints} using really simple but incredibly fast algorithms. We can also use an intelligent use of colour. \\

\noindent \textbf{Physical Constraints} industrial vision is usually solved by applying simple computer vision algorithms, and lots of physical contraints.
\begin{itemize}
    \item Environment: lighting, enclosure, mounting
    \item Acquisition hardware: expensive camera, optics, filters
\end{itemize}

