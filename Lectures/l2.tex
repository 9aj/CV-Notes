\section{Image Sampling}

\subsection{Sampling Signals}
Given a continuos signal, an image is well sampled if from the sample points alone, an accurate representation of the original signal can be created. Bad sampling would describe a different signal upon creation.
\\
If we were to oversample, lets use the example of a rotating wheel, the wheel would appear to be rotating slowly. If the sampling rate was not high enough, the wheel would be appear to be moving in the opposite direction.
\\
Consider the frequency domain, the spectra repeat, if the sampling rate is correct, the spectra will just touch. We can deduce from this that:
\begin{equation}
    Sample_{min} = 2 \times f_{max}
\end{equation}
This follows \textit{Nyquists sampling theorem}, which states:
\begin{quote}
    In order to be able to reconstruct a singal from its samples, we must sample at minimum twice the maximum frequency in the original sample.
\end{quote}
A simple example of this would be, speech averages at $6kHz$, we sample at $12kHz$. If we compare this to something like \textit{.wav} audio files, they are often sampled at 24kHz upon a spectral analysis. In image terms; two pixels for every pixel of interest.

\subsection{1D Discrete Fourier Transform}
Discrete Fourier calculates frequency from data points.

\begin{align}
    \mathcal{F}p(\omega) = \int_{-\infty}^{\infty}p(t)e^{-jwt}dt \\
    \mathcal{F}p_{u} =\frac{1}{N}\sum_{i=0}^{N-1}p_{i}e^{-j\frac{2\pi}{N}iu} \\
\end{align}

Where:
\begin{align}
    \text{Sampled Frequency: } \mathcal{F}p_{u} \\
    \text{Sampled Points: } p_{i}\\
    N \text{ points}
\end{align}

Again, we can reconstruct these signals by adding all six frequency components:



